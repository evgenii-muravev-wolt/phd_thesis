\chapter{Introduction}
Life needs energy to continue its spread. Plants use photosynthesis to separate carbon from oxygen and to grow. Higher life forms as humans consume energy during the day and during the night. While fossil fuels are the major source of energy and fire is used to convert the Joules per mol of hydrocarbons into kilowatt-hours in the power socket, there are cleaner and more efficient ways to harvest energy. Photosynthesis has inspired the creation of solar plants, the atom had been tamed in the core of a nuclear reactor, we use the energy of the wind and waves; the thermal energy of the Earth and there is a hope for the nuclear fusion. Nowadays, all means of pulling the energy to the consumer aim at the electrical energy that is stored in form of electric charges separated by a potential barrier. The storage of electrical energy is the key ingredient of every power grid, every electrical device. It is convenient to store the electrical energy in an electrochemical cell, that is, to store the opposite charges on two spatially separated materials that have different work functions, or, chemically speaking, oxidation potentials. 